%commands
% Eigene Befehle und typographische Auszeichnungen für diese


% Abkürzungen mit korrektem Leerraum 
\newcommand{\ua}{\mbox{u.\,a.}\xspace}
\newcommand{\zB}{\mbox{z.\,B.}\xspace}
\newcommand{\zb}{\mbox{z.\,B.}\xspace}
\newcommand{\Zb}{\mbox{Z.\,B.}\xspace}
\newcommand{\dahe}{\mbox{d.\,h.}\xspace}
\newcommand{\Vgl}{Vgl.\xspace}
\newcommand{\bzw}{bzw.\xspace}
\newcommand{\evtl}{evtl.\xspace}

\newcommand{\abbildung}[1]{Abbildung~\ref{fig:#1}\xspace}

% zum Ausgeben von Autoren
\newcommand{\AutorName}[1]{\textsc{#1}}
\newcommand{\Autor}[1]{\AutorName{\citeauthor{#1}}}


\newcommand{\TDD}{\ac{tdd}\xspace}
\newcommand{\BDD}{\ac{bdd}\xspace}

\newcommand{\todo}[1]{
  \ifdraft{\marginpar{\textcolor{red}{\texttt{ #1}}}}
  {}
}

\newcommand{\footurl}[1]{\footnote{\url{#1}}}

%manuelle Trennung
%\hyphenation{update-\_multiple}

%% better: (general command to vertically center horizontal material)
\newcommand*{\vcenteredhbox}[1]{\begingroup\setbox0=\hbox{#1}\parbox{\wd0}{\box0}\endgroup}
\newcommand*{\vcenteredvbox}[1]{\begingroup\setbox0=\vbox{#1}\parbox{\wd0}{\box0}\endgroup}

\newcommand*{\lsttt}[1]{\lstinline[basicstyle=\ttfamily\color{black}\footnotesize, columns=fixed]!#1!}
\newcommand*{\smalltexttt}[1]{{\ttfamily\footnotesize #1} }

\lstnewenvironment{ruby}[1][]
{\lstset{
    basicstyle=\ttfamily\color{black}\scriptsize,
    identifierstyle=\color{colIdentifier},
    keywordstyle=\color{colKeys}\bfseries,
    stringstyle=\color{colString},
    commentstyle=\color{colComments},
    columns=fixed,
    tabsize=2,
    frame=none,
    extendedchars=true,
    showspaces=false,
    showstringspaces=false,
    numbers=none,
    numberstyle=\tiny,
    breaklines=true,
    breakautoindent=true,
    language=Ruby,
    morekeywords={test, require, it},
    morestring=[b]/,
    emph={Given, And, When, Then},
    emphstyle=\color{colConstant}\bfseries,
    #1}}
{}

\lstdefinelanguage{Cucumber}{
    morekeywords={Szenario, Funktionalität, Angenommen, Und, Wenn, Dann, Scenario, Feature, Given, And, When, Then},
    morecomment=[l]{\#},
    morestring=[b]"
%   morestring=[b]'
}

\lstnewenvironment{cucumber}[1][]
{\lstset{
    basicstyle=\ttfamily\color{black}\scriptsize,
    identifierstyle=\color{colIdentifier},
    keywordstyle=\color{colKeys}\bfseries,
    stringstyle=\color{colString},
    commentstyle=\color{colComments},
    columns=fixed,
    tabsize=2,
    frame=none,
    extendedchars=true,
    showspaces=false,
    showstringspaces=false,
    numbers=none,
    numberstyle=\tiny,
    breaklines=true,
    breakautoindent=true,
        language=Cucumber,
        #1}
}{}