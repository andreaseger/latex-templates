% Packages

% Deutsche Anpassungen
\usepackage{babel}
\usepackage{scrhack}
\usepackage{ifdraft}

% Schrift

% Font
\ifXeTeX
  \usepackage{fontspec}
  \usepackage{xunicode}
  \usepackage{xltxtra}
  \setmainfont{Minion Pro}
  \setsansfont{Myriad Pro}
  \setmonofont{Consolas}
\else
  \ifLuaTeX
    \usemodel[simplefonts]
    \setmainfont[Minion Pro]
    \setmonofont[Consolas]
%    \setmathfont[mymathfont]
    \setsansfont[Myriad Pro]
    \enabletrackers[fonts.missing]
  \else% pdftex
    \usepackage[T1]{fontenc}
    % \usepackage[garamond]{mathdesign}
    % \usepackage[scaled]{helvet}
    % \usepackage{MinionPro}
    % \renewcommand{\sfdefault}{Myriad-LF}
    % \usepackage[scaled]{beramono}
  \fi
\fi

% bessere typography
\ifXeTeX
  \usepackage{microtype}
\else
  \usepackage[protrusion=true, expansion=true, final]{microtype}
\fi

% 1,5 facher Zeilenabstand in draft
\ifdraft{
\usepackage[onehalfspacing]{setspace}
}{
\usepackage{setspace}
}

%dictum
\renewcommand{\dictumwidth}{.5\textwidth}

% bessere typography
%\usepackage[protrusion=true, expansion=true, final]{microtype}

%dynamic space
\usepackage{xspace}

% Grafiken
\usepackage{graphicx}
%\usepackage{epstopdf}
\graphicspath{{img/}}
\usepackage{framed}

%wasserzeichen
\usepackage{eso-pic}
\makeatletter
\newcommand\BackgroundPicture[3]{%
	\setlength{\unitlength}{1pt}%
	\put(0,\strip@pt\paperheight){%
		\parbox[t][\paperheight]{\paperwidth}{%
			\vfill
			\centering\includegraphics[width=#2,angle=#3]{#1}
			\vfill
		}
	}
} %
\makeatother




% Zum einbinden der Titelseite
\usepackage{pdfpages}

%% Glossar
\usepackage[printonlyused, nohyperlinks]{acronym}
\def\bflabel#1{{\textbf{#1}\dotfill}}

% zum Einbinden von Programmcode -----------------------------------------------
\usepackage[final]{listings}
\usepackage{xcolor}
%\definecolor{lightgrey}{RGB}{216,220,214}

\ifdraft{
\definecolor{colKeys}{rgb}{0,0,0}
\definecolor{colIdentifier}{rgb}{0,0,0}
\definecolor{colComments}{rgb}{0,0,0}
\definecolor{colString}{rgb}{0,0,0}
\definecolor{colConstant}{RGB}{0,0,0}
}{
\definecolor{colKeys}{RGB}{225,119,1}
\definecolor{colIdentifier}{rgb}{0,0,0}
\definecolor{colComments}{RGB}{83,92,103}
\definecolor{colString}{rgb}{0,0.5,0}
\definecolor{colConstant}{RGB}{190,1,25}
}


\lstnewenvironment{ruby}[1][]
{\lstset{
    basicstyle=\ttfamily\color{black}\scriptsize,
    identifierstyle=\color{colIdentifier},
    keywordstyle=\color{colKeys}\bfseries,
    stringstyle=\color{colString},
    commentstyle=\color{colComments},
    columns=fixed,
    tabsize=2,
    frame=none,
    extendedchars=true,
    showspaces=false,
    showstringspaces=false,
    numbers=none,
    numberstyle=\tiny,
    breaklines=true,
    breakautoindent=true,
		language=Ruby,
		morekeywords={test, require, it},
		morestring=[b]/,
		emph={Given, And, When, Then},
		emphstyle=\color{colConstant}\bfseries,
		#1}}
{}

\lstdefinelanguage{Cucumber}{
	morekeywords={Szenario, Funktionalität, Angenommen, Und, Wenn, Dann, Scenario, Feature, Given, And, When, Then},
	morecomment=[l]{\#},
	morestring=[b]"
%	morestring=[b]'
}

\lstnewenvironment{cucumber}[1][]
{\lstset{
    basicstyle=\ttfamily\color{black}\scriptsize,
    identifierstyle=\color{colIdentifier},
    keywordstyle=\color{colKeys}\bfseries,
    stringstyle=\color{colString},
    commentstyle=\color{colComments},
    columns=fixed,
    tabsize=2,
    frame=none,
    extendedchars=true,
    showspaces=false,
    showstringspaces=false,
    numbers=none,
    numberstyle=\tiny,
    breaklines=true,
    breakautoindent=true,
		language=Cucumber,
		#1}
}{}

% Bibliographiestil
\usepackage{natbib}

% urls
\usepackage[hyphens]{url}
%% Define a new 'leo' style for the package that will use a smaller font.
\makeatletter
\def\url@leostyle{%
  \@ifundefined{selectfont}{\def\UrlFont{\sf}}{\def\UrlFont{\footnotesize\ttfamily}}}
\makeatother
%% Now actually use the newly defined style.
\urlstyle{leo}


% Querverweise
\usepackage[
    plainpages=false,
    unicode=true,          % non-Latin characters in Acrobat?s bookmarks
    pdftoolbar=true,        % show Acrobat?s toolbar?
    pdfmenubar=true,        % show Acrobat?s menu?
    pdffitwindow=true,     % window fit to page when opened
    pdfstartview={FitV},    % fits the width of the page to the window
    pdftitle={\titel},    % title
    pdfauthor={\autor},     % author
    pdfsubject={\art},   % subject of the document
    pdfkeywords={\titel, \untertitel}, % list of keywords
    pdfnewwindow=true,      % links in new window
    colorlinks=false,       % false: boxed links; true: colored links
    linkcolor=red,          % color of internal links
    citecolor=green,        % color of links to bibliography
    filecolor=magenta,      % color of file links
    urlcolor=cyan,          % color of external links
    hyperfootnotes=false,
 							]{hyperref}
\usepackage[all]{hypcap}


% Kopf und Fußzeile
\usepackage[automark]{scrpage2}
\clearscrheadings
%\clearscrplain
%\clearscrheadfoot

\setkomafont{pageheadfoot}{\small\sffamily}
\setkomafont{pagenumber}{\normalfont\normalsize}
\chead{\headmark} % Oben innen: Setzt Kapitel
\cfoot{} % Mitte unten: Seitenzahlen bei plain
\ofoot{\pagemark}
\ifoot{\autor}
\pagestyle{scrheadings}

%\usepackage{chapterthumb}
%\renewcommand*{\chapterthumbskip}{.16\paperheight}%